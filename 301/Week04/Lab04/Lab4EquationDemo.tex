\documentclass{article}

\begin{document}

\LaTeX \ equations are easy to write, and usually self-explanatory. An inline equation starts and ends with a \$. Like this: $y=f(x)$.

Display equations need a double \$\$, like this:

$$ y=x^2 $$

See if you can find how to do "$x$ is square root of $y$" as a display equation?

Here it is (thanks Bao!), and here's the more standard way to write display equations:

\begin{equation}
x = \sqrt{y}
\end{equation}

Another one:  summation and subscripts

\begin{equation}
x = \sum_i z_i
\end{equation}

Try to change that to "$x=$ sum of $z$ from $i=1$ to $n$".

\begin{equation}
x = \sum_{i=1}^n z_i
\end{equation}

If your sub/super-script is $>1$ characters long, make sure to enclose it with curly brackets.  Talking about brackets... here's how they work in an equation

\begin{equation}
x = \left(\sum_{i=1}^n z_i\right)^2
\end{equation}

Compare to this:

\begin{equation}
x \neq \sum_{i=1}^n \left(z_i\right)^2
\end{equation}

You have to use the "left" and "right" commands, to make the auto-sizing work.

Try to do some of this with an integral on x from ... to ...

$$ x \geq \int_{0}^{\infty} g(t)dt $$

Now, some Greek and fractions:

\begin{equation}
\bar{x} = \frac{1}{n}\sum_{i=1}^n x_i
\end{equation}

Let's put text \emph{inside} math, like this $x\leq y\mathrm{\ and \ } x \geq z$.

Where's the Greek? You find it. Write an equation E[x] = mu.

$$ E[x] = \mu.$$

Capital Greek, and the probability keyword:

$$ \Pr\left(Z\leq z\right) = \Phi(z) $$

...finally, equation arrays:

\begin{equation}
\begin{array}{rcl}
	y & = & \mathbf{X}\beta+\epsilon \\
	\epsilon & \sim & N\left(0,\sigma^2\right)
\end{array}
\end{equation}


\end{document}